\documentclass[10pt,a4paper]{jarticle}
\usepackage{docmute}
\usepackage{tc2016_utf}
\usepackage[dvipdfmx]{graphicx,color}
\usepackage[fleqn]{amsmath}
\usepackage{algorithm,algorithmic}
\usepackage{amssymb,epsfig}
\usepackage{ascmac}

\usepackage{url}
\usepackage{bm}
\usepackage{ascmac}
\usepackage{pifont}
%\usepackage{multirow}
\usepackage{enumerate}
%\usepackage{cases}
\usepackage{type1cm}
\usepackage{here}
\usepackage{secdot}
\sectiondot{subsection}
\sectiondot{subsubsection}

\input{jdummy.def}

\def\vec#1{\mbox{\boldmath$#1$}}
\def\vector#1{\mbox{\boldmath $#1$}}

\newcommand{\argmax}{\mathop{\rm arg~max}\limits}
\newcommand{\argmin}{\mathop{\rm arg~min}\limits}
\newcommand{\umax}{\mathop{\rm max}\limits}

\def\R{{\Bbb R}}
\def\Z{{\Bbb Z}}

\renewcommand{\topfraction}{0.8}
\renewcommand{\bottomfraction}{0.8}
\renewcommand{\dbltopfraction}{0.8}
\renewcommand{\textfraction}{0.1}
\renewcommand{\floatpagefraction}{0.8}
\renewcommand{\dblfloatpagefraction}{0.8}
\setcounter{topnumber}{3}
\setcounter{bottomnumber}{3}
\setcounter{totalnumber}{3}
\begin{document}
\section{つくばチャレンジ2016の結果と考察}
\subsection{実験走行}
横断歩道には挑戦しなかった.これは我々のチームがこれまで完走を達成できたことが無かったため,まずは完走を目指したためである.
中央公園後半付近ではランドマークになるものが少なく,上手く地図作成できず,実験走行では自律走行に失敗することがあった.

\subsection{本走行結果}
本走行の結果,横断歩道を除く課題コースの全区間を完走することができた.また,人物探索を行うような走行経路を通ったが人物探索は行っていない.実験走行で上手く機能しないことが判明したためである.
不安定だった中央公園の後半付近の地図を本走行の前に取り直すことで,本走行では完走を達成できた.一方で,オドメトリの向上を行い,ランドマークが無い環境でも安定して地図作成ができるような仕組みを取り入れる必要があると考えられる.


\end{document}

% Local Variables:
% mode: yatex
% TeX-master: "tc2016_third"
% End: