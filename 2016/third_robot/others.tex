\documentclass[10pt,a4paper]{jarticle}
\usepackage{docmute}
\usepackage{tc2016_utf}
\usepackage[dvipdfmx]{graphicx,color}
\usepackage[fleqn]{amsmath}
\usepackage{algorithm,algorithmic}
\usepackage{amssymb,epsfig}
\usepackage{ascmac}

\usepackage{url}
\usepackage{bm}
\usepackage{ascmac}
\usepackage{pifont}
%\usepackage{multirow}
\usepackage{enumerate}
%\usepackage{cases}
\usepackage{type1cm}
\usepackage{here}
\usepackage{secdot}
\sectiondot{subsection}
\sectiondot{subsubsection}

\input{jdummy.def}

\def\vec#1{\mbox{\boldmath$#1$}}
\def\vector#1{\mbox{\boldmath $#1$}}

\newcommand{\argmax}{\mathop{\rm arg~max}\limits}
\newcommand{\argmin}{\mathop{\rm arg~min}\limits}
\newcommand{\umax}{\mathop{\rm max}\limits}

\def\R{{\Bbb R}}
\def\Z{{\Bbb Z}}

\renewcommand{\topfraction}{0.8}
\renewcommand{\bottomfraction}{0.8}
\renewcommand{\dbltopfraction}{0.8}
\renewcommand{\textfraction}{0.1}
\renewcommand{\floatpagefraction}{0.8}
\renewcommand{\dblfloatpagefraction}{0.8}
\setcounter{topnumber}{3}
\setcounter{bottomnumber}{3}
\setcounter{totalnumber}{3}
\begin{document}

\section{開発状況などに関して}
本年度の開発では,昨年度までに開発してきたロボットを用いた.開発を行うにあたり,我々のチームは実験走行になかなか参加出来ないため,学内での実験を積み重ねてきた.学内の実験においても約1000[m]程度の自律走行が安定してできていた.しかし,学内での実験では他のロボットや歩行者などの未知の障害物などが少なく,それらを想定した実験を行う必要があった.

\section{つくばでの実験の流れ}
\begin{description}
 \item[10月31日]\mbox{}\\
	    ロボットやテントなどを梱包
 \item[11月01日]\mbox{}\\
	    ロボットや機材などを搬送業者に引き渡し
 \item[11月03日]\mbox{}\\
	    つくば市着 ロボットの準備
 \item[11月04日]\mbox{}\\
	    第6回実験走行 ロボットの安全確認,大清水公園のデータ取得,大清水公園の地図作成,ウェイポイントの設定,大清水公園の自律走行,確認走行,課題コース全体のデータ取得を行った
 \item[11月05日]\mbox{}\\
	    第7回実験走行 課題コース全体のテスト走行,探索対象の発見及びアプローチの実験,遠隔監視システムの動作確認
 \item[11月06日]\mbox{}\\
	    本走行 朝の走行実験で第7回実験走行で上手く自律走行出来なかった箇所の地図を再度取得.ウェイポイントの再設定.交流会終了後,ロボット等の梱包.
 \item[11月07日]\mbox{}\\
	    北九州着
\end{description}
つくばチャレンジに遠方から参加する場合,なかなか実験走行に参加できないため,実験走行日にどれだけ実験できるかどうかが重要になってくる.

\section{最後に}
我々が開発したソースコードは,GitHub上で\url{https://github.com/CIR-KIT-Unit03}や\url{https://github.com/CIR-KIT}として公開している.
開発を行うにあたって参考にしていただければありがたい.


\section*{謝辞}
つくばチャレンジ実行委員会やつくば市の方々にはつくばチャレンジのような貴重な実験の機会を与えていただき感謝いたします.
\end{document}

% Local Variables:
% mode: yatex
% TeX-master: "tc2016_third"
% End: