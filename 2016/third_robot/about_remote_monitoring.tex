\documentclass[10pt,a4paper]{jarticle}
\usepackage{docmute}
\usepackage{tc2016_utf}
\usepackage[dvipdfmx]{graphicx,color}
\usepackage[fleqn]{amsmath}
\usepackage{algorithm,algorithmic}
\usepackage{amssymb,epsfig}
\usepackage{ascmac}

\usepackage{url}
\usepackage{bm}
\usepackage{ascmac}
\usepackage{pifont}
%\usepackage{multirow}
\usepackage{enumerate}
%\usepackage{cases}
\usepackage{type1cm}
\usepackage{here}
\usepackage{secdot}
\sectiondot{subsection}
\sectiondot{subsubsection}

\input{jdummy.def}

\def\vec#1{\mbox{\boldmath$#1$}}
\def\vector#1{\mbox{\boldmath $#1$}}

\newcommand{\argmax}{\mathop{\rm arg~max}\limits}
\newcommand{\argmin}{\mathop{\rm arg~min}\limits}
\newcommand{\umax}{\mathop{\rm max}\limits}

\def\R{{\Bbb R}}
\def\Z{{\Bbb Z}}

\renewcommand{\topfraction}{0.8}
\renewcommand{\bottomfraction}{0.8}
\renewcommand{\dbltopfraction}{0.8}
\renewcommand{\textfraction}{0.1}
\renewcommand{\floatpagefraction}{0.8}
\renewcommand{\dblfloatpagefraction}{0.8}
\setcounter{topnumber}{3}
\setcounter{bottomnumber}{3}
\setcounter{totalnumber}{3}
\begin{document}
\section{遠隔監視システム}
つくばチャレンジの注意事項等について,「ロボットの位置や状況のステーションにおけるモニタリング」を強く推奨すると明記されている.我々は本要求事項を達成する遠隔監視システムを構築したので,その機能とネットワーク環境について述べる.

初めに,遠隔監視について述べる.監視画面として,ロボットの現在姿勢を地図上にリアルタイムに表示させる形式を採用した.現在姿勢,あるいは現在姿勢と経路履歴の両方,を選択的に表示できるようにした.今回の実験走行時に現在姿勢の全履歴を表示した様子をFig.\ref{monitor}に示す.スタートからゴールまでの経路が表示されていることが確認できる.加えて,探索対象アプローチ位置も同時に表示できるようにした.

\begin{figure}
    \centering
    \includegraphics[width=8cm]{fig/png/monitor.png}
    \caption{遠隔監視システムで地図上に現在姿勢を表示させた様子.}
    \label{monitor}
\end{figure}

次に,通信環境について述べる.監視端末とロボット間の通信回線として商用モバイル回線を利用し,VPNで接続を行った.VPNシステムとして,OpenVPN \cite{openvpn} を採用した.遠隔監視PCをサーバ,ロボットをクライアントとしてVPNネットワークを構成し,ROS ネットワークと連携させることで,遠隔監視を実現した.

工夫点としては,ロボットが一定距離を移動する毎に,自身の位置を監視端末に送信する仕様とした点が挙げられる.モバイル回線でOpenVPNを利用するという制約があるため,通信容量を節減することが目的である.今回の実験走行では送信間隔を5[m]に設定したところ,通信容量を大幅に低減することに成功した.

\end{document}

% Local Variables:
% mode: yatex
% TeX-master: "tc2016_third"
% End:
