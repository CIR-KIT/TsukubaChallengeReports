\documentclass[10pt,a4paper]{jarticle}
\usepackage{docmute}
\usepackage{tc2016_utf}
\usepackage[dvipdfmx]{graphicx,color}
\usepackage[fleqn]{amsmath}
\usepackage{algorithm,algorithmic}
\usepackage{amssymb,epsfig}
\usepackage{ascmac}

\usepackage{url}
\usepackage{bm}
\usepackage{ascmac}
\usepackage{pifont}
%\usepackage{multirow}
\usepackage{enumerate}
%\usepackage{cases}
\usepackage{type1cm}
\usepackage{here}
\usepackage{secdot}
\sectiondot{subsection}
\sectiondot{subsubsection}

\input{jdummy.def}

\def\vec#1{\mbox{\boldmath$#1$}}
\def\vector#1{\mbox{\boldmath $#1$}}

\newcommand{\argmax}{\mathop{\rm arg~max}\limits}
\newcommand{\argmin}{\mathop{\rm arg~min}\limits}
\newcommand{\umax}{\mathop{\rm max}\limits}

\def\R{{\Bbb R}}
\def\Z{{\Bbb Z}}

\renewcommand{\topfraction}{0.8}
\renewcommand{\bottomfraction}{0.8}
\renewcommand{\dbltopfraction}{0.8}
\renewcommand{\textfraction}{0.1}
\renewcommand{\floatpagefraction}{0.8}
\renewcommand{\dblfloatpagefraction}{0.8}
\setcounter{topnumber}{3}
\setcounter{bottomnumber}{3}
\setcounter{totalnumber}{3}
\begin{document}
\section{遠隔監視システム}
つくばチャレンジの注意事項等について,「ロボットの位置や状況のステーションにおけるモニタリング」を強く推奨すると明記されている.我々は本要求事項を達成する遠隔監視システムを構築したので,その機能とネットワーク環境について述べる.
ただし,本システムは大会当日に機能させることが出来なかったため,以下では開発時の情報を用いて説明する.

初めに,遠隔監視について述べる.監視画面として,ロボットの現在姿勢を地図上に表示させる形式を採用した.現在姿勢,あるいは現在姿勢と経路履歴の両方,を選択的に表示できるようにした.現在姿勢を表示した様子をFig.\ref{monitor}に示す.なお,同図で画像の一部のみを拡大をしている箇所は説明のために加工を施したもので,実際のシステムでは背景の大域地図上にロボットの姿勢が表示されるのみである.
\begin{figure}
    \centering
    \includegraphics[width=6cm]{fig/png/monitor.png}
    \caption{遠隔監視システムで地図上に現在姿勢を表示させた様子.}
    \label{monitor}
\end{figure}

次に,通信環境について述べる.監視端末とロボット間の通信回線として商用モバイル回線を利用し,VPNで接続を行った.VPNシステムとして,OpenVPNを採用した.遠隔監視PCをサーバ,ロボットをクライアントとしてVPNネットワークを構成し,ROS ネットワークと連携させることで,遠隔監視を実現した.

ここで,工夫点について記述する.本システム開発当初,本構成でROSの可視化ツールであるrvizによる監視を試みたところ,rvizはリアルタイムに更新される位置情報を全て取得しようと試みるのだが,商用回線の通信速度ではその情報量に対応しきれず,情報の更新漏れが頻繁に発生してしまった.また,パケット通信容量が膨大で,契約した通信回線の上限を容易に超えてしまう恐れがあり,遠隔監視そのものが実現できなくなる懸念があった.
そこで,ロボットが一定距離を移動する毎に,自身の位置を監視端末に送信する仕様に変更することで,更新漏れとパケット通信料の問題を解決した.検証段階では送信間隔5[m]毎に設定したが,ロボットの位置を追跡するには十分な周期であった.

最後に,本機能の展望を記述する.ネットワークシステムを独自で構築したため,位置に限らず任意の情報を遠隔監視するための基盤技術を確立できた.期待される機能の例として,ロボットの現在速度,電池残量,検出した人物の画像等といった,遠隔監視に有益な情報表示機能が挙げられる.


\end{document}

% Local Variables:
% mode: yatex
% TeX-master: "tc2016_third"
% End: